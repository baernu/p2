\section{Analyse}
Ziel dieser Arbeit war es, dem Betrachter ein virtuelles Fenster zu suggerieren. Dieses Fenster sollte sich also genauso verhalten wie ein reales Fenster. \\ Bei dieser Umsetzung zentral ist  unser optisches Sehen, welches auf perpektivischem Sehen beruht. Dieses perspektivische Sehen galt es genau so umzusetzen. Dies wird in Unity von der Render Kamera gelöst, welche ein perspektivisches Bild ausgibt. Das ist in dieser Arbeit also sehr gut umgesetzt. \\ Der etwas schwierigere Teil war es, im Bildschirm  nur einen Teil von diesem Bild darzustellen. Dazu dient der Screen in der virtuellen Szene um zu bestimmen, was genau auf dem  Bildschirm dargestellt werden soll. Diesen Bildschirm genau zu erfassen erfordert ein genaues Wissen vom intrinsischen und extrinsischen Verhalten der virtuellen Render-Kamera und deren Position und Ausrichtung. \\
Daraus lässt sich dann eine Matrix berechnen um im perspektivischen Bild der Render-Kamera die Eckpunkte vom Screen zu bestimmen. \\ Diesen Teil habe ich in meiner Arbeit nicht so umsetzen können. Ich habe das etwas vereinfacht. Mein Ansatz geht in die gleiche Richtung aber so, dass es sich in Unity möglichst einfach umsetzen lässt. In Unity mit der Methode ViewportToWorldPoint können die Eckpunkte vom geklippten Bild zurückgegeben werden. Ausgehend von diesen Punkten konstruiere ich diese Ebene und projeziere dann perspektivisch den Screen auf diese Ebene. \\ Dies funktioniert soweit gut, einzig wenn die Kamera nicht ganz horizontal ausgerichtet ist, bekomme ich in meiner geklippten Ebene für den horizontalen Richtungsvektor auch Werte für die y-Komponente die nun nicht mehr Null sind. \\ Bei der Bestimmung der relativen Anteile jedes Eckpunktes bezüglich der Richtungsvektoren bekomme ich zwei Gleichungen mit zwei Unbekannten. Damit das ganze etwas einfacher wird, setzte ich bei dem horizontalen Richtungsvektor die y-Komponente gleich Null. Im Endbild sieht man daher ganz aussen an den Rändern, dass die Eckpunkte vom Screen minim nicht ganz richtig gewählt wurden. Da mir nun aber die Zeit fehlt belasse ich dies so. \\ Insgesamt denke ich, dass der Eindruck vom virtuellen Fenster recht gut vermittelt wird. Die Auflösung ist in etwa auch so, dass ein Bildschirm in der Pixelgrösse 1200 x 900 ein akzeptables Bild zeigt. Die Anpassungsgeschwindigkeit an neue Positionen seitens Betrachter erscheint ohne grosse Hacker. \\ Insgesamt erzielt diese Arbeit ein gutes Resultat, so meine Einschätzung.