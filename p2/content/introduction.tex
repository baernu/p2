\section{Introduction}

Bilder faszinieren mich schon sehr lange. Anfangs waren es Gemälde, Zeichnungen, Skizzen, analoge Fotografien, später dann digitale Bilder und bewegte Bilder. 
Ich bin nun in der Vertiefung Computer Perception and Virtual Reality. Im Gespr\"ach mit dem Dozenten Marcus Hudritsch kamen wir auf die Thematik von Raum- und Farbempfinden im Zusammenhang mit Augmented Reality. Es ging darum ein Projektthema zu finden in dieser Vertiefung der Informatikausbildung.
Marcus Hudritsch hat mir dann von dieser Umsetzung erzählt: Ein Bildschirm welcher ein künstliches Fenster in eine virtuelle Welt suggeriert.
Ich stelle mir vor der Betrachter steht etwas verwirrt vor diesem Screen und begreift erst allm\"ahlich. Zuerst begreift er, dass sich diese Szene dargestellt auf dem Screen bewegt, dann nach einigem hin und her gehen, dass sich die Szene mit ihm im Gleichschritt bewegt und dann nach genauem Beobachten, wie sich die virtuellen Objekte verändern, dass die perspektivische Darstellung stimmt und es ist als schaue er durch ein Fenster in eine künstliche Welt, von Objekten. \\ Details werde ich in dieser Arbeit später aufzeigen. Wichtig ist mir hier deutlich zu machen, dass es in dieser Umsetzung auch darum geht, den Betrachter von der Realität in die Virtualität hinein zu begeleiten und etwas ins Grübeln zu bringen. \\ Dies wird nur gelingen wenn die Illusion von einem Fenster unserem gewohnten Empfinden gerecht wird und es sich als realtime Simulation anfühlt.
