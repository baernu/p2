
In meiner Projektarbeit möchte ich einen tieferen Einblick erhalten wie Augmented Reality an einem eigentlich einfachen Anwendungsfall umgesetzt werden kann.\\ Stellen Sie sich folgende Szene vor: In einem Raum hängt ein grösserer Bildschirm. Eine Tiefenkamera, die gerade oberhalb dem Bildschirm montiert ist, trackt den Betrachter, sobald sich dieser innnerhalb dem getrackten Bereich befindet. Die Tiefenkamera ist vom Typ Intel Realsense Kamera D435. \\ In einer virutellen Szene, welche mittels der Gameengine Unity dargestellt wird, wird die Position von der getrackten Person auf eine virutelle Kamera gemappt. Der Blick der virtuellen Kamera entspricht dem Blick des Betrachters. In dieser Szene wird nun ein virtuelles Fenster dargestellt. Dieses Fenster wird perspektivisch in 2D dargestellt und erscheint als verzerrtes Rechteck - je nach Betrachtungswinkel. Nun soll dieses Polygon entzerrt und dann auf dem Bildschirm vor dem Betrachter dargestellt werden. \\ Dem Betrachter erscheint so auf dem Bildschirm ein virtuelles Fenster, deren sichtbare Objekte sich mit der Position des Betrachters perspektivisch verschieben. Der Betrachter, schaut in den Bildschirm und erlebt eine virtuelle Szene, welche sich so verhält, wie wenn er durch ein reales Fenster in die reale Welt blickt. \\ Dieses Empfinden des Betrachters versuche ich mit dieser Arbeit zu erreichen.

