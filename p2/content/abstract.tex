
In meiner Projektarbeit m\"ochte ich einen tieferen Einblick erhalten wie Augmented Reality an einem eigentlich einfachen Anwendungsfall umgesetzt werden kann.\\ Der Betrachter welcher vor einem Bildschirm steht, wird von einer Intel Realsense Kamera D435 erfasst und dessen Position relativ zur Kamera wird ermittelt. In einer virutellen Szene, welche mittels der Gameengine Unity dargestellt wird, wird die Position von der getrackten Person auf eine virutelle Kamera gemappt. Der Blick der virtuellen Kamera entspricht dem Blick des Betrachters. In dieser Szene wird nun ein virtuelles Fenster dargestellt. Dieses Fenster wird perspektivisch in 2D dargestellt und erscheint als verzerrtes Rechteck - je nach Betrachtungswinkel. Nun soll dieses Polygon entzerrt und dann auf dem Bildschirm vor dem Betrachter dargestellt werden. \\ Dem Betrachter erscheint so auf dem Bildschirm ein virtuelles Fenster, deren sichtbare Objekte sich mit der Position des Betrachters perspektivisch verschieben. Der Betrachter soll sich visuell vor einem Fenster in eine virtuelle Welt empfinden.

